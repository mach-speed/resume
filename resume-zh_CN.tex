% !TEX TS-program = xelatex
% !TEX encoding = UTF-8 Unicode
% !Mode:: "TeX:UTF-8"

\documentclass{resume}
\usepackage{zh_CN-Adobefonts_external} % Simplified Chinese Support using external fonts (./fonts/zh_CN-Adobe/)
% \usepackage{NotoSansSC_external}
% \usepackage{NotoSerifCJKsc_external}
% \usepackage{zh_CN-Adobefonts_internal} % Simplified Chinese Support using system fonts
\usepackage{linespacing_fix} % disable extra space before next section
\usepackage{cite}

\begin{document}
\pagenumbering{gobble} % suppress displaying page number

\name{李启航}

\basicInfo{
  \email{974558391@qq.com} \textperiodcentered\ 
  \phone{(+86) 188-4559-9348} \textperiodcentered\ 
  \textsection{出生年月: 1998.12}
  }
 
\section{\faGraduationCap\  教育背景}
\datedsubsection{\textbf{哈尔滨工业大学}\space 本科,计算机科学}{2016.09 -- 2020.06}
\begin{onehalfspacing}
学院排名约30$\%$。计算理论/算法/CSAPP/数学公共课/离散数学等课程排名前15$\%$。

\end{onehalfspacing}

\section{\faUsers\ 工作经历}

\begin{onehalfspacing}
\datedsubsection{\textbf{阿里巴巴}\space 智能引擎(前搜索推荐事业部)-预测引擎。开发工程师}{2021.08 - 至今}
\begin{itemize}
  \item HA3 SQL样本实验开发,通过实时展开星形模型,实现了训练任务ad-hoc查询多个表组合的样本,用跨样本共享物理表的方式节省了相当多的存储成本(在做列实验的场合)
  \item 实现了序列join功能,将存储占比最大的序列特征,每一组只保留序列id,该组序列特征用右表维表和序列id结合补全。工作内容包括python sdk开发,calicte模块扩展(rel2sql和sql2rel),meta管理。
  \item 样本平台升级接入管控平台的新方案,实现了集群拉起,和数据表的构建周期自动管理。在接入期间对管控平台的许多功能进行了各项完善。
  \item 样本平台开始接入各个业务使用后,承担了大约40$\%$的答疑和运维工作。包括发现各种内部组件的瓶颈,推进bug发现和修复。
\end{itemize}
\end{onehalfspacing}

\begin{onehalfspacing}
\datedsubsection{\textbf{字节跳动}\space   互娱研发-数据策略/Growth-数据应用。大数据工程师}{2020.07 - 2021.08}
\begin{itemize}
  \item 抖音/火山/头条/电商的活动/策略/算法相关数据开发。四人小组峰值CPU核数60k+
  \item 数据可视化平台Go后端开发,实现了各类BI需求,技术选型有预计算和ClickHouse聚合。
  \item 在组内负责大部分的实时特征工作,比较复杂的有使用CoProcessFunction拼接AB流和事件流,相比此前普遍使用的Hive方案,AB数据反馈周期从3小时以上,降低到分钟级;实时累计窗口(用户使用时长);近期滑动窗口。
\end{itemize}
\end{onehalfspacing}


\section{\faCogs\ 技术栈}
% increase linespacing [parsep=0.5ex]
\begin{itemize}[parsep=0.5ex]
  \item 工作范围内,实际有用过的语言有Java,Python,Scala,Go。数据库有用过Mysql,MongoDB,Redis。可视化平台,管控平台(既有样本meta,也有更基础的离线表,在线表,集群,拉起,sync,等管控),等都有实际开发的经验。工作中会用到的C++开发的组件,有一定debug能力。
  \item 大数据相关Flink了解比较多,理解原理和概念。Hive/Spark/ClickHouse/Hologres/Ha3都有过使用经验(做过基于这些技术选型的方案,Hologres/Ha3是在阿里期间做的样本平台的数据链路实现)。接触过一点calcite优化的概念,有做过样本平台的table api(relnode树到sql)和sql parser。有扩展过sql 方言。
  \item Emacs, Git, ssh, screen, perf, docker,等开发和运维工具,有一定的掌握和了解,有在自己工作范围内的运维经验。
  \item 日常使用 Linux,熟悉UNIX/Linux环境。近期有读过计算机组成原理-软硬件接口/量化研究方法。有学过Rust语言。
  \item 对机器学习有基础的了解,工作都算是和泛机器学习平台相关的。用户遇到的问题能帮助debug一部分。
  \item 大一过四六级。有熟练的文档阅读能力。

  
\end{itemize}

\section{\faInfo\ 自我评价}
% increase linespacing [parsep=0.5ex]
\begin{onehalfspacing}
对计算机科学本身有热情。工作有责任感,能投入。做事有始有终。重视业务和技术价值的统一。
\end{onehalfspacing}

%% Reference
%\newpage
%\bibliographystyle{IEEETran}
%\bibliography{mycite}
\end{document}
