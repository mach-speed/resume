% !TEX TS-program = xelatex
% !TEX encoding = UTF-8 Unicode
% !Mode:: "TeX:UTF-8"

\documentclass{resume}
\usepackage{zh_CN-Adobefonts_external} % Simplified Chinese Support using external fonts (./fonts/zh_CN-Adobe/)
% \usepackage{NotoSansSC_external}
% \usepackage{NotoSerifCJKsc_external}
% \usepackage{zh_CN-Adobefonts_internal} % Simplified Chinese Support using system fonts
\usepackage{linespacing_fix} % disable extra space before next section
\usepackage{cite}

\begin{document}
\pagenumbering{gobble} % suppress displaying page number

\name{李启航}

\basicInfo{
  \email{974558391@qq.com} \textperiodcentered\ 
  \phone{(+86) 137-1825-7093} \textperiodcentered\ 
  \textsection{出生年月: 1998.12}
  }
 
\section{\faGraduationCap\  教育背景}
\datedsubsection{\textbf{哈尔滨工业大学}\space 本科,计算机科学}{2016.09 -- 2020.06}
\begin{onehalfspacing}
学院排名约30$\%$。
对编程语言和计算模型感兴趣。Scheme/Erlang/Haskell等函数式语言有了解。
计算理论/算法/CSAPP/数学公共课/离散数学等课程排名前15$\%$。

\end{onehalfspacing}

\section{\faUsers\ 工作经历}
\begin{onehalfspacing}
\datedsubsection{\textbf{字节跳动}\space IES数据策略(现Growth增长智能-抖音),大数据工程师}{2020.07 - 2021.04}
\begin{itemize}
  \item 抖音火山的活动,策略,算法相关数据支持。承担春节期间抖火主会场50$\%$以上指标开发,及方案验证。四人小组峰值CPU核数60k+。
  \item 数据可视化平台Golang后端开发,实现了各类BI需求。在数据链路上验证了ClickHouse方案的可用性(近似聚合,分片,抽样,BloomFilter等feature)。
  \item 实时AB数据开发,使用CoProcessFunction拼接AB流和事件流(low level join)。相比此前普遍使用的Hive方案,AB数据反馈最小周期从3小时以上,降低到分钟级。
  \item 抖音极速版的特征和样本开发。包括实时特征计算和实时样本拼接。
\end{itemize}
\end{onehalfspacing}

\begin{onehalfspacing}
\datedsubsection{\textbf{字节跳动}\space Growth数据应用,大数据工程师}{2021.05 - 至今}
\begin{itemize}
  \item Growth的特征开发。承担了头条极速版和电商等业务主要的实时特征工作。
  \item 实现了实时累计用户使用时长等较复杂的特征需求。
\end{itemize}
\end{onehalfspacing}


\section{\faCogs\ 技术栈}
% increase linespacing [parsep=0.5ex]
\begin{itemize}[parsep=0.5ex]
  \item Flink,在小组内对流式计算应用,学习最多,有Debug经验,理解Metric和Web UI中各项数据含义。读过Stream System,及Flink主要相关论文(Dataflow,异步快照,MillWheel等)。
  \item Hive/Spark/ClickHouse有较多使用经验。小组内承担复杂Scala Spark任务迁移到SQL的需求。
  \item 熟悉Scala,Java,Python的Numpy/Pandas计算库。
  \item 对ML了解在同级生平均水平以上。在校做过基于Tensorflow的简单项目。有ML场景下数据链路工作经验。
  \item 大一过四六级。有熟练的文档阅读能力。

  
\end{itemize}

\section{\faInfo\ 自我评价}
% increase linespacing [parsep=0.5ex]
\begin{onehalfspacing}
对分布式计算相关领域有热情。
追求业务价值和技术进步的统一。
工作有责任感能投入。
\end{onehalfspacing}

%% Reference
%\newpage
%\bibliographystyle{IEEETran}
%\bibliography{mycite}
\end{document}
